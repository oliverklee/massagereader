\chapter{Grundsätzliches}

\section{Raum und Atmosphäre}

\subsection{Keine Störungen}

Sorgt dafür, dass ihr beim Massieren nicht gestört werdet: Schaltet das Telefon aus oder legt es in einen anderen Raum, sagt den MitbewohnerInnen Bescheid, oder hängt ein Nicht-stören-Schild an die Tür.

\subsection{Temperatur}

Der Raum sollte warm sein, und es sollte nicht ziehen. 28~Grad sind eine richtig gute Temperatur.

\subsection{Licht}

Sorgt für gedämpftes, gemütliches Licht. Kerzen, Salzsteinlampen und Lichterketten sind prima. Ich selbst benutze Philipps Hue mit einem entspannenden Farbschema. Neonröhren hingegen gehen gar nicht.

\subsection{Musik}
Die Musik sollte den Charakter der Massage unterstützen und nicht zu sehr ablenken.

\subsubsection{Entspannungsmassagen}
Für eine möglichst entspannende Massage sind Meditationsmusik, Gregorianik oder andere sehr ruhige Musik sehr gut. Ich benutze dafür meist Deuter, David Hykes, Enya oder das Hilliard Ensemble.

\subsubsection{Massagen mit mehr Energie}
Wenn die Massage anregen soll (einfach so, oder im erotischen Sinne) empfehle ich Musik von Loreena McKennitt, Jan Garbarek, Yanni, Andreas Vollenweider oder das Album \emph{Drum Sex} von Brent Lewis.


\section{Massageöl}

\subsection{Ein Öl auswählen}

Am wichtigsten ist, dass das Massageöl auf pflanzlichen Ölen basiert und keine Mineralöle enthält. Süßes Mandelöl aus dem Bioladen oder aus der Apotheke ist prima. Das aromatisierte Weleda-Massageöl ist auch prima. Die meisten Babyöle taugen nichts, weil sie aus Mineralölen bestehen.

Ihr könnt auch in der Küche nach Ölen suchen: Olivenöl ist gut, riecht aber etwas intensiv. Sonnenblumenöl ist sehr fett und ergiebig.

\subsection{Keine Ölkatastrophen}

Sinnvoll ist ein Teller oder Tablett als Unterlage fürs Öl gegen Ölkatastrophen sowie Taschentücher oder Küchentücher zum Ab- und Aufwischen. Hilfreich ist auch, wenn die Ölflasche eine kleine Öffnung hat, sodass das Öl nicht ausläuft, wenn die Ölflasche umkippt.

\subsection{Öl vorher anwärmen}

Wenn das Öl zu kalt (beispielsweise, weil ihr es wie ich im Kühlschrank aufbewahrt) auf die Haut kommt, verkrampfen die Muskeln, die ihr eigentlich entspannen wollt. Deswegen sollte das Öl schön warm sein.

Dafür könnt ihr das Öl auf der Heizung oder in einer Tasse mit warmem Wasser vorher auf Körpertemperatur bringen.


\section{Kommunikation}

\subsection{Vor der Massage}

Sprecht vorher ab, was eure Partnerin massiert haben möchte, welche Stellen besonderen Massagebedarf haben und welche vielleicht kitzlig oder besonders empfindlich sind. Fragt, wie viel Druck und \glqq gute Schmerzen\grqq\  (\glqq Wohlweh\grqq) eure Partnerin aushält und haben möchte. Hier sind die Unterschiede sehr groß: Manche mögen nur etwas stärkere Streichungen, und andere genießen die guten Schmerzen sehr.

\subsection{Während der Massage}

Ihr könnte auch absprechen, wie viel und welches Feedback ihr während der Massage bekommt:

\subsubsection{Weniger Feedback, mehr Genuss}
Wohliges Schnurren, Brummen und Grunzen, wenn es besonders schön ist, und ein \emph{Aua!}, wenn es wehtut. Diese Variante erhöht den Genuss (weil eure Partnerin sich besser fallen lassen kann), reduziert aber den Lerneffekt für euch. Ich empfehle diese Variante, wenn ihr schon mehr Übung habt oder eure Partnerin einfach nur genießen möchte.

\subsubsection{Mehr Feedback, weniger Genuss}
Eure Partnerin gibt euch während der Massage (zusätzlich zum wohligen Brummen) verbales Feedback, beispielsweise wenn ihr stärker oder weniger stark massieren könntet, wenn ein Griff etwas unangenehm ist oder eine Streichung etwas länger sein sollte. Diese Variante verringert den Genuss (weil eure Partnerin die ganze Zeit mit dem Kopf dabei ist), ist für euch aber zum Lernen extrem hilfreich. Vor allem für den Anfang empfehle ich diese Variante.

\subsection{Nach der Massage}

Holt euch nach der Massage Feedback: Was war besonders schön? Was kann ich noch verbessern?


\section{Lagerung}

\subsection{Die Unterlage}

Die Unterlage sollte weich genug sein, dass eure Partnerin liegen kann, ohne irgendwo unangenehmen Druck zu spüren, aber auch hart genug sein, damit ihr Druck auf den Körper ausüben könnt, ohne eure Partnerin in die Unterlage hineinzudrücken. Außerdem braucht ihr genug Platz, um links und rechts neben dem Körper knien oder stehen zu können. Außerdem sollte die Unterlage etwas Öl vertragen können.

Die meisten Betten sind zum Massieren zu weich. Ein hartes Futon (mit oder ohne Bettdecke) ist gut. Ihr könnt auch eine Isomatte auf den Boden legen und mit einem weichen Badetuch abdecken. Zusätzlich könnt ihr die Unterlage noch weicher gestalten, indem ihr Decken zwischen Isomatte und Badetuch legt.


\subsection{Lücken füllen}

Die Bauchlage kann eine Handtuchrolle unter den Fußgelenken bequemer machen.

Bei der Rückenlage könnt ihr ein flaches Kissen unter den Kopf legen und bei Bedarf eine Handtuchrolle unter die Kniekehle.

Generell gilt, dass ihr bei einer härteren Unterlage die Lücken eher füllen wollt als bei einer etwas weicheren Unterlage.


\subsection{Die Arme lagern}

Eure Partnerin sollte die Arme am besten längs neben ihren Körper legen. Wenn sie die Arme nach oben nimmt, spannt sie damit den Trapezmuskel zwischen Schulter und Nacken an, sodass ihr diesen Muskel dann nicht gut massieren könnt.


\subsection{Den Kopf lagern}

Optimal ist ein Massagetisch mit einem Loch für den Kopf, weil eure Partnerin dann den Kopf gerade liegen hat. Hilfsweise könnt ihr auch aus einer Handtuchwurst ein umgekehrtes \emph{U} formen, in dessen Kuhle eure Partnerin dann ihren Kopf legt.

Oder sie legt den Kopf etwas zur Seite. Dann solltet ihr beachten, dass ihr beim oberen Rücken nur die Rückenhälfte (links/rechts) gut massieren könnt, auf den die Nase gerade nicht zeigt. Bittet bei Bedarf eure Partnerin, ihren Kopf zu wenden.


\subsection{Warm halten}

Sorgt dafür, dass eure Partnerin nicht friert. Deckt die Körperteile, die ihr gerade nicht massiert, mit einer dünnen Decke oder einem Handtuch ab. Bettdecken sind dafür etwas zu dick; ich habe gute Erfahrungen mit Fleecedecken gemacht.


\section{Die Masseurin}

\subsection{Bequeme Kleidung}

Tragt beim Massieren bequeme Kleidung, die euch nicht einengt.

\subsection{Hände}

Legt vorher eure Ringe ab, weil diese kratzen können und außerdem ölig werden. Eure Fingernägel sollten kurz sein, damit ihr kneten könnt. Wascht euch vor dem Massieren die Hände.

\subsection{Bequem sitzen}

Wenn ihr seitlich arbeitet, könnt ihr neben eurer Partnerin knien (die Japaner nennen das \emph{Seiza}).

Wenn ihr mittig arbeitet, könnt ihr über den Oberschenkeln eurer Partnerin knien (nicht \emph{auf} den Beinen oder dem Po, weil das eurer Partnerin wehtun kann) oder am Kopfende im \emph{Seiza} knien.

Wechselt öfter mal die Position, damit ihr besser an die Stellen herankommt, die ihr gerade massiert, und damit ihr keine Rückenschmerzen bekommt.

Muskelkater in den inneren oberen Oberschenkelmuskeln ist übrigens völlig normal, wenn ihr sonst selten massiert.


\section{Vor der Massage}

\subsection{Zur Ruhe kommen}

Versucht, vor der Massage selbst zur Ruhe zu kommen, weil ihr eure Ruhe oder Unruhe beim Massieren auf eure Partnerin übertragt. Eine angenehme Atmosphäre (s.\,o.) kann euch dabei helfen. Trinkt vor der Massage vielleicht noch eine nette Tasse Tee zusammen.


\subsection{Die Hände aufwärmen}

Wärmt eure Hände (also eure Massagewerkzeuge) vor dem Massieren auf:

\begin{itemize}
  \item Handgelenke kreisen
  \item Handgelenke nach vorne und hinten dehnen
  \item kräftig in die Luft greifen (oder eine Decke, einen Igelball oder Tennisball)
  \item Handflächen gegeneinander reiben
\end{itemize}


\section{Die Ölflasche aufschrauben}

Öffnet spätestens zu diesem Zeitpunkt die Ölflasche. Später wird es sehr schwierig, die Flasche zu öffnen, ohne den Kontakt zu brechen (s.\,u.).

\section{Kontakt aufnehmen}

Legt am Anfang der Massage eure Hände auf und nehmt erst einmal Kontakt mit eurer Partnerin auf.


\section{Während der Massage}

Während der Massage solltet ihr ein paar grundsätzliche Dinge beachten:

\begin{itemize}
  \item Versucht, währen der Massage die ganze Zeit mit mindestens einer Hand Kontakt zur Partnerin zu halten.
  \item Gießt oder tropft das Massageöl niemals direkt auf eure Partnerin, weil das eure Partnerin überraschen könnte und weil das Öl möglicherweise kalt ist. Bildet stattdessen mit einer Hand eine Schale, gießt dort etwas Öl hinein, verreibt das Öl in beiden Händen, und verstreicht dann mit euren Handflächen das Öl auf eurer Partnerin. Haltet auch hier den Kontakt mit eurer Partnerin.
  \item Eure Partnerin sollte während der Massage die Augen geschlossen halten. Das macht die Massage für euch beide deutlich entspannter und entspannender.
\end{itemize}


\section{Nach der Massage}

\subsection{Verabschieden}

Macht noch ein paar abschließende Streichungen mit wenig Druck.

Lasst eure Hände noch etwas liegen und fühlt die Verbindung. Übertragt Energie, wenn ihr daran glaubt. Nehmt nach einiger Zeit eure Hände sacht weg.


\subsection{Hände reinigen}

Schüttelt eure Hände kräftig aus. Falls ihr die Möglichkeit habt, wascht euch zusätzlich die Hände.


\subsection{Feedback holen}

Holt euch Feedback (s.\,o.).

\subsection{Das Lächeln eurer Partnerin genießen}

Genug gesagt.
