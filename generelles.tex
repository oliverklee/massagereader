\section{Grundsätzliches zu Massage}

\subsection{Raum und Atmosphäre}
\begin{enumerate}
  \item warm, kein Zug
  \item Musik:
    \begin{itemize}
      \item nicht ablenkende, Ruhe verbreitende Musik für entspannende, meditative Massage: Deuter, David Hykes, einiges von Yanni \ldots
      \item Musik mit etwas mehr Energie für intensive Massage, die sich am Rhythmus der Musik orientiert: Enya, Andreas Vollenweider, Jan Garbarek, einiges von Yanni, diverse Lounge-Sachen \ldots
    \end{itemize}
  \item keine Störungen:
    \begin{itemize}
      \item Telefon abstellen/leise stellen
      \item \emph{Nicht stören!}-Schild an die Tür hängen
      \item den MitbewohnerInnen Bescheid sagen
    \end{itemize}
\end{enumerate}
    
\subsection{Zubehör}
\begin{enumerate}
  \item Massageöl:
    \begin{itemize}
      \item kein Babyöl, da dieses aus Mineralöl besteht und die Haut austrocknet
      \item Mandelöl (süß oder kaltgepresst) ist relativ billig, fasst geruchslos und sehr gut für die Haut
      \item Öl aus der Küche geht auch gut: Sonneblumenöl, Distelöl, Olivenöl (ist allerdings nicht ganz geruchsneutral)
      \item Eventuell kann man ätherische Öle zumischen. Dabei sollte man auf Allergien achten (zum Beispiel gegen Orangenöl).
    \end{itemize}
  \item Sinnvoll: Unterlage für Öl, Taschentücher o.~Ä.~zum Aufwischen
  \item Alternative Massagegeräte: Igel, Rollen, Tennisbälle \ldots
\end{enumerate}

\subsection{Lagerung}
\begin{enumerate}
\item Unterlage zum Drauflegen, nicht zu weich
\item Kissen für unter Kopf, Bauch und Beine
\item Kopf am besten gerade, wenn es noch bequem ist
\item Eventuell Handtuchwurst als umgekehrtes \emph{U} mit Mulde für den Kopf
\item T-Shirt, Decke, großes Kissen o.~Ä.~zum Abdecken gerade nicht massierter Partien
\end{enumerate}

\subsection{Masseur/Masseurin}

\subsubsection{Allgemeines}
\begin{enumerate}
\item Bequeme Kleidung
\item Keine Ringe, Halsketten o.~Ä.
\item Kurze Fingernägel
\item Vorher Hände waschen!
\end{enumerate}

\subsubsection{Bequem sitzen}
\begin{enumerate}
\item Daneben, über Beinen knien, am Kopfende knien
\item Öfter die Position wechseln (dann kommt man auch besser dran)
\end{enumerate}

\subsubsection{Hände vorbereiten}
\begin{enumerate}
\item Handgelenke kreisen
\item Fingerspitzen gegeneinander
\item Gummiball oder Igel drücken
\item Hände selbst kurz massieren (vor allem die Finger)
\end{enumerate}

\subsection{Vor der Massage}
\begin{enumerate}
\item Selbst zur Ruhe kommen
\item Kontakt herstellen (Hände auflegen), eventuell in gleichen Atemrhythmus kommen
\item Verspannungen der Haut prüfen (zupfen)
\end{enumerate}

\subsection{Während der Massage}
\begin{enumerate}
\item Nicht beide Hände gleichzeitig wegnehmen
\item Streichungen am \emph{ganzen} Rücken (ganz unten bis ganz oben)
\item Öl nicht direkt auf Rücken, sondern in Hand $\Rightarrow$ wärmt an
\item Kein Druck auf die Wirbelsäule!!!
\item Nur wenig Druck in Nierengegend!!!
\item Die Muskeln massieren, nicht die Knochen!
\item Erst flachere Schichten (Streichungen etc.), dann mitteltiefe, dann tiefere (Knetungen etc.)
\item Feedback geben lassen!!!
\end{enumerate}

\subsection{Nach der Massage}
\begin{enumerate}
\item Noch ein paar Streichungen
\item Verspannungen prüfen
\item Hände noch liegenlassen, Kontakt halten, evtl.~Energiefluss spüren
\end{enumerate}
