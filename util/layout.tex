%----------------------------------------------------------------------------------
% Pakete und Parameter
%----------------------------------------------------------------------------------

% Input-Encoding für UTF-8
\usepackage[utf8]{inputenc}

% Grafikpaket
\usepackage{color}
\usepackage[pdftex]{graphicx}

% relativer Pfad zu den Bildern
\graphicspath{{../images/}}

% Absätze werden nicht eingezogen, sondern vertikal abgesetzt
\usepackage{packages/noindent}

% Palatino und Helvetica statt Computer Modern als Standard-Fonts
\usepackage{palatino}

% Bibliographieeinstellungen
\usepackage{natbib}
\bibliographystyle{alpha}

% lesbare Verweise
\usepackage[pdftex,plainpages=false,pdfpagelabels]{hyperref}

% Fonts in PDF hübsch machen
\usepackage{ae}

% nette URLs
\usepackage{url}

% Deutsch als Dokumentsprache
\usepackage[ngerman]{babel}

% für schattierte, ovale Boxen etc.
\usepackage{fancybox}
\usepackage{framed}
\definecolor{shadecolor}{rgb}{0.8,0.8,0.8}

%----------------------------------------------------------------------------------
% Seitenlayout
%----------------------------------------------------------------------------------

% Seiten-Kopfzeilen und -Fußzeilen
\usepackage{fancyhdr}
\pagestyle{fancy}
\fancyhf{}
\fancyhead[RE]{\slshape \nouppercase{\leftmark}}    % links:  "Seite      Kapitel"
\fancyhead[LO]{\slshape \nouppercase{\rightmark}}   % rechts: "Kapitel    Seite"
\fancyhead[RO,LE]{\bfseries \thepage}
\renewcommand{\headrulewidth}{1pt} % Kopfzeilen unterstreichen
\renewcommand{\footrulewidth}{0pt}

\fancypagestyle{plain}{ % Keine Kapitel und Abschnitt auf Startseite start pages
\fancyhf{}
\fancyhead[RO,LE]{\bfseries \thepage}
\renewcommand{\headrulewidth}{1pt}
\renewcommand{\footrulewidth}{0pt}
}

% Kopfzeile auf linker Seite: "1  Einführung"
\renewcommand{\chaptermark}[1]{%
\markboth{\thechapter\ \ \ \ #1}{}}

% Kopfzeile auf rechter Seite: "1.1  Basics"
\renewcommand{\sectionmark}[1]{%
\markright{\thesection\ \ \ \ #1}{}}

% Seitenlayout
\topmargin20mm
\addtolength{\headheight}{2pt} % verhindert zu volle vboxes durch fancyhdr
\footskip10mm % Abstand von unserem Rand zu Datum

\newlength{\fullwidth} % Seites des Textes plus Randnotizen
\setlength{\fullwidth}{\textwidth}
\addtolength{\fullwidth}{\marginparsep}
\addtolength{\fullwidth}{\marginparwidth}

% Maximale Gliederungstiefe, die noch ins Inhaltsverzeichnis aufgenommen wird
\setcounter{tocdepth}{1}

% zweispaltiges Layout möglich machen
\usepackage{multicol}

\raggedbottom
