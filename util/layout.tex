%----------------------------------------------------------------------------------
% Pakete und Parameter
%----------------------------------------------------------------------------------

% Sonderzeichen fixen
\usepackage[T1]{fontenc}

% Input-Encoding für UTF-8
\usepackage[utf8]{inputenc}

% Grafikpaket
\usepackage{xcolor}
\usepackage[pdftex]{graphicx}

% relativer Pfad zu den Bildern
\graphicspath{{../images/}}

% Absätze werden nicht eingezogen, sondern vertikal abgesetzt
\usepackage{packages/noindent}

% Palatino und Helvetica statt Computer Modern als Standard-Fonts
\usepackage{palatino}

% Bibliographieeinstellungen
\usepackage{natbib}
\bibliographystyle{alpha}

% lesbare Verweise
\usepackage[pdftex,plainpages=false,pdfpagelabels]{hyperref}

% nette URLs
\usepackage{url}

% Deutsch als Dokumentsprache
\usepackage[ngerman]{babel}

% Anführungszeichen sprachabhängig machen
\usepackage[babel]{csquotes}

% für Boxen etc.
\usepackage{framed}
\definecolor{shadecolor}{rgb}{0.8,0.8,0.8}


%----------------------------------------------------------------------------------
% Seitenlayout
%----------------------------------------------------------------------------------

% Seiten-Kopfzeilen und -Fußzeilen
\usepackage{scrlayer-scrpage}

\iftoggle{long}{
  % Kopfzeile auf linker Seite: "1  Einführung"
  \renewcommand{\chaptermark}[1]{%
  \markboth{\thechapter\ \ \ \ #1}{}}

  % Kopfzeile auf rechter Seite: "1.1  Basics"
  \renewcommand{\sectionmark}[1]{%
  \markright{\thesection\ \ \ \ #1}{}}
}

% Seitenlayout
\topmargin20mm
\footskip10mm % Abstand von unserem Rand zu Datum

\newlength{\fullwidth} % Seites des Textes plus Randnotizen
\setlength{\fullwidth}{\textwidth}
\addtolength{\fullwidth}{\marginparsep}
\addtolength{\fullwidth}{\marginparwidth}

% Maximale Gliederungstiefe, die noch ins Inhaltsverzeichnis aufgenommen wird
\setcounter{tocdepth}{1}

% zweispaltiges Layout möglich machen
\usepackage{multicol}

% Descriptions ohne Einzug
\renewenvironment{description}[1][0pt]
{\list{}{
    \labelwidth=0pt \leftmargin=#1
     \let\makelabel\descriptionlabel
  }
}
{\endlist}

\raggedbottom
