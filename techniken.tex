\chapter{Liste der Techniken}

\paragraph{Ausstreichen:} Streicht mit beiden Händen abwechselnd in dieselbe Richtung.

\paragraph{Bahnen:} Einen Technik auf einer langen Strecke mehrfach parallel oder abwechselnd ausführen.

\paragraph{Beißen:} Versucht dabei, die Muskeln zu bewegen, anstatt Löcher zu machen.

\paragraph{Daumendruck-Kreise:} Beide Daumen liegen parallel und machen mit Druck kleine Kreise. Verschiebt nach und nach die Daumen parallel ohne Druck.

\paragraph{Daumenstreichungen:} Auf der Stelle mit beiden Daumen abwechselnd schräg in V-Form nach oben. Beide Daumen bearbeiten dabei denselben Bereich.

\paragraph{Druckmassage:} Punktueller Druck auf der Stelle. Meist mit den Daumen.

\paragraph{Endlos-Techniken:} Dies sind Streichungen mit beiden Händen in derselben Richtung. Beide Hände sind dabei an gegenüberliegenden Stellen des Weges und kreuzen sich in der Mitte, sodass die Illusion einer unendlichen Bewegung entsteht.

\paragraph{Fingerkreise:} Die Fingerspitzen beschreiben kleine Kreise. Dabei können die Kreise entweder auf der Stelle bleiben oder sich verschieben.

\paragraph{Fingernagel-Streichen:} Die Finger sind zur Hälfte angewinkelt, sodass ihr eine halbe Faust habt. Streicht so mit der Handflächenseite der halben Fäuste, dass ihr gegen die Kratzrichtung streicht.

\paragraph{Fingerstreichungen:} Mit den Fingerspitzen streichen. Eure Handflächen dienen dabei zur Stabilisierung.

\paragraph{Greifen:} Kneten, wobei die Daumen an den Fingern anliegen und ihr den Handballen benutzt.

\paragraph{Hacken:} Mit der linken und rechten Handkante abwechselnd. Das Handgelenk ist dabei locker.

\paragraph{Kneten:} Knetet die Muskeln wie einen Kuchenteig, indem ihr mit den Fingern kräftig hineingreift. Der Daumen kann dabei den Fingern gegenüber sein oder auf derselben Seite (dann ist der Handballen der Gegenpunkt).

\paragraph{Milchtritt:} Beide Hände auflegen, sodass eure Hände sich leicht überlappen. Greift mit den Fingerspitzen kräftig in die Muskeln, ohne die Hände zu heben.

\paragraph{Ölverteilen:} Gebt das Öl in eure Hände und verteilt es mit Streichungen mit dem Handflächen (bei großen Flächen) oder den Fingern.

\paragraph{Rubbeln:} Beide Handflächen liegen mit den Daumenseiten nebeneinander. Verschiebt die Hände gegeneinander hin und her.

\paragraph{Streichungen:} Die Hand oder Finger über eine Fläche bewegen. Streichungen können auf der Stelle sein (Fingerstreichungen), aber auch so lang wie ein Rücken oder Bein sein.

\paragraph{Streichungen mit Fingerflächen:} Beide Hände liegen aufeinander. Die Handballen zeigen nach unten. Macht kleine Bewegungen in dieselbe Richtung.

\paragraph{Zickzack-Kneten:} Die Daumen sind abgespreizt. Beide Hände berühren sich mit den Daumen. Wringt die Muskeln aus. Beide Hände bewegen sich dabei gegenläufig.

\paragraph{Ziehen (nach außen):} Beide Hände liegen nebeneinander auf und ziehen voneinander weg.

\paragraph{Ziehen (nach oben):} Beide Hände sind dabei auf gegenüberliegenden Seiten des Körpers. Die Finger zeigen in Richtung Unterlage. Zieht dann mit beiden Händen in Richtung Körperachse. An der Körperachse können sich die Hände.

