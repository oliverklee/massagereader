\iftoggle{long}{
  \section{Rücken und Gesäß}
}{
  \section{Rücken}
}

\subsection{Generelles}

\iftoggle{long}{
  Wenn ihr mehrere Körperteile massiert, ist die Rückenmassage ein guter Anfang: Die meisten Menschen sind dort verspannt, und dies ist ein relativ wenig intimer Körperbereich. Außerdem kann man dort wenig kaputtmachen, was es eurer Partnerin einfacher macht, sich zu entspannen, falls ihr euch noch nicht so gut kennt.

  Vor allem die Streichungen könnt ihr sehr \fett{oft wiederholen}.
}{}

Eure Partnerin legt bei der Rückenmassage ihre Arme am besten parallel zum Körper auf die Unterlage.


\subsubsection{Wo massieren und wo besser nicht?}

\paragraph{Wirbelsäule:} Drückt \emph{niemals} direkt auf die Wirbelsäule, weil das sehr schmerzhaft ist.

\paragraph{Nieren:} Seid vorsichtig mit Druck in der Nierengegend, weil das schmerzhaft ist.

\iftoggle{long}{
  \paragraph{Knochen, Sehnen, Nerven:} Einige Stellen am Körper werdet ihr nicht weichmassieren können. Die meisten dieser Stellen werden als \emph{Knochen} bezeichnet, und die anderen nennt man \emph{Sehnen}.

  Außerdem gibt es am Rücken einige Nervenpunkte, an denen starker Druck Schmerzen verursacht (die unangenehme Sorte Schmerzen). Hört auf eure Partnerin, die euch sagen wird, wenn es wehtut.
}{}


\iftoggle{long}{
  \subsubsection{Reihenfolge}
  Der Rücken hat mehrere Muskelschichten. Massiert dabei von außen nach innen~-- also zuerst mit Streichungen die oberen Muskelschichten, und arbeitet euch dann nach und nach weiter nach innen zum Kneten. Falls ihr direkt mit tieferen Griffen anfangt, würdet ihr dabei durch verspannte Muskelschichten durchmassieren müssen, was für eure Partnerin schmerzhaft ist und für euch anstrengend.

  Achtet darauf, dass ihr beide Rückenhälften gleichermaßen massiert. Ansonsten fühlt sich die Massage unvollständig an.

  \subsubsection{Wie und wo sitzen}

  Wenn ihr gerade Streichungen und Griffe benutzt, könnt ihr über den Oberschenkeln eurer Partnerin knien. Setzt euch aber dabei nicht auf ihre Beine oder den Po, weil das an den Knien oder der Hüfte für sie schmerzhaft sein kann. Manche Griffe könnt ihr auch von oben ausführen, indem ihr am Kopfende kniet.

  Für Griffe von der Seite kniet ihr neben eurer Partnerin~-- je nach Griff eher rechtwinklig zu ihrem Rücken oder eher schräg.

  Wenn ihr mit Kraft massieren möchtet, geht das mit schiebenden Bewegungen am besten. Beim Ziehen habt ihr kaum Kraft.

  Wenn ihr mit Druck massieren möchtet, verschiebt euren Schwerpunkt über eure Arme. Dadurch benötigt ihr deutlich weniger Kraft, als wenn ihr nur aus den Armen heraus massieren würdet.


  \subsubsection{Geschwindigkeit}
  Für eine entspannende Massage massiert mit etwas langsamer Streichungen, als eure Partnerin atmet. Für eine energetisierende Massage massiert ihr etwas schneller.


  \subsubsection{Größe der Massagezone}

  Der Rücken fängt unten am oberen Rand des Becken an und endet oben, links und rechts dort, wo der Torso eurer Partnerin die Unterlage berührt. Massiert bei Streichungen die komplette Fläche, damit sich die Massage vollständig anfühlt.

  Ausnahmen sind die Brüste eurer Partnerin (abhängig davon, wie nah ihr euch seid) und Stellen, an denen eure Partnerin sehr kitzlig ist.
}{}


\iftoggle{long}{
  \pagebreak
}{}
\subsection{Der Einstieg}

\subsubsection{Kontakt aufnehmen}
\fett{Legt eure Hände in der Nierengegend auf} und nehmt erst einmal Kontakt mit eurer Partnerin auf. Ihr könnt dabei im gleichen Rhythmus atmen, wenn ihr möchtet und wenn euch das nicht zu intim ist, oder Energie übertragen, wenn ihr daran glaubt.

\subsubsection{Die Lage sondieren}
Geht mit euren Händen über den Rücken, drückt hier und da ein wenig und und \fett{spürt, wo die Verspannungen sitzen} und wie sich die Haut anfühlt.

\subsubsection{Das Öl verteilen}

Gebt \fett{Öl in eine Hand} und verreibt das Öl dann in beiden Händen. Haltet dabei mit einer Hand den Kontakt zu eurer Partnerin. \fett{Verteilt das Öl} mit langen Streichungen gleichmäßig auf dem Rücken.


\pagebreak
\subsection{Streichungen vom Becken zum Kopf}
Hierfür kniet ihr über den Oberschenkeln eurer Partnerin. Diese Streichungen könnt ihr auch zusätzlich vom Kopfende aus machen (dann mit Druck in Richtung Becken).

Alle diese Streichungen führt ihr mit Kraft vom Becken in Richtung Kopf aus. Für den Rückweg streicht ihr mit wenig Druck auf dem Rücken oder den Seiten entlang.

Diese Griffe sind \fett{symmetrisch}, wobei die Wirbelsäule eurer Partnerin die gedachte Spiegelachse ist.

\begin{enumerate}
  \item \fett{Hände parallel:} Handflächen parallel links und rechts neben die Wirbelsäule legen mit den Fingern in Richtung Kopf
  \item \fett{Hände quer:} Hände links und rechts neben die Wirbelsäule legen mit den Fingern in Richtung Wirbelsäule
  \item \fett{Tannenbaum:} Daumen rechtwinklig zur Wirbelsäule, Finger schräg nach oben, sodass Finger und Daumen ein Dreieck bilden
  \item \fett{V:} Daumen parallel neben die Wirbelsäule aufsetzen, Finger zu je einem V zum Daumen nach außen legen
  \item \fett{Fäuste:} Setzt eure Fäuste mit der "`Schlagfläche"' neben die Wirbelsäule und schiebt nach oben. Benutzt oben an den Schulterblättern nur ganz wenig Druck.
  \iftoggle{long}{
    \item \fett{Alma-Move:} Formt eure Hände zu Krallen und setzte die Fingerspitzen jeder Hand parallel links und rechts neben die Wirbelsäule. Geht mit Druck nach oben, dann weiter zwischen den Schulterblättern, und streicht an den Seiten mit flachen Händen zurück nach unten. Zieht das Becken leicht nach unten in Richtung der Füße.
  }{}
\end{enumerate}

\iftoggle{long}{
  \subsection{Querstreichungen mit der Handfläche}

  Auch hier kniet ihr über den Oberschenkeln eurer Partnerin.

  Ihr streicht dabei mit beiden Handflächen \fett{von der Wirbelsäule quer nach außen}. Die Finger zeigen dabei nach außen, und der Druck kommt der kompletten Handfläche. Arbeitet euch so ihn Bahnen von der Hüfte bis zum Kopf hoch. Haltet dabei immer mit mindestens einer Hand Kontakt.
}{}

\iftoggle{long}{
  \pagebreak
}{}
\subsection{Mitteltiefes im rechten Winkel von der Seite}
Hierfür sitzt ihr \fett{rechtwinklig neben eurer Partnerin}, führt alle Streichungen aus, und geht dann auf die andere Seite, um die Streichungen noch einmal von der anderen Seite auszuführen.

\begin{oframed}
  \fett{Von jeder Seite:}
  \begin{enumerate}
    \item \fett{Nach oben ziehen:} je eine Handfläche an jeder Rückenseite, dann nach innen ziehen und die Hände kreuzen lassen
    \iftoggle{long}{
      \item \fett{Rücken öffnen 1:} Legt eure beiden Hände in "`Bet-Haltung"' mit den Handkanten quer zur Wirbelsäule auf die Mitte des Rückens. Geht dann zuerst mit euren Handkanten nach außen, sodass eure Handflächen nach unten zeigen. Zieht sehr langsam mit Kraft nach außen oben bis zum Nacken und unten bis zum Kreuzbein. Haltet diese Position ein paar Sekunden. Achtet darauf, dass ihr keinen Druck direkt auf die Wirbelsäule ausübt.
      \item \fett{Rücken öffnen 2:} Legt eure Unterarme quer zur Wirbelsäule auf die Mitte des Rückens. Die Bewegung ist dann dieselbe wie beim vorherigen Schritt.
    }{}
    \item \fett{Schwingungen:} mit den Handballen die gegenüberliegende Hüfte und untere Rückenhälfte zum Schwingen bringen
    \item \fett{Außenseiten kneten:} die gegenüberliegen Außenseite zwischen Fingern und Daumen greifen, leicht im Zickzack kneten oder wringen und dabei vom Becken in Richtung Schulter wandern und wieder zurück
    \item \fett{Rückenstrecker mit Daumenstreichungen:} auf der gegenüberliegenden Seite der Wirbelsäule die Muskeln mit sehr kleinen Daumenstreichungen mit viel Druck massieren
    \iftoggle{long}{
      \item \fett{Endlos-Streichungen auf dem Rückenstrecker mit Fingerknöcheln:} Formt Tigerkrallen mit euren Händen. Streicht dann mit dem zweiten Knöchel der Finger beider Hände den Rückenstrecker der \fett{gegenüberliegenden Rückenhälfte} von der Hüfte bis zum Kopf aus. Stützt bei Bedarf euren Zeigefinger mit dem Daumen.
    }{}
  \end{enumerate}
\end{oframed}


\pagebreak
\subsection{Wenn das Öl schon etwas eingezogen ist}

Hierbei \fett{kniet ihr zuerst über den Beinen} eurer Partnerin und \fett{massiert von unten}.

\iftoggle{long}{
  Diese Griffe bringt ihr am besten dann unter, wenn das Öl schon etwas eingezogen ist und die Haut griffiger ist~-- wann genau, ist nicht so wichtig.
}{}

\begin{enumerate}
  \item \fett{Rubbeln:} Legt die Finger beider Hände flach direkt nebeneinander auf den Rücken, sodass die Finger in Richtung Kopf zeigen. Beide Hände sind dabei auf \emph{derselben} Rückenhälfte. Rubbelt gegenläufig auf und ab, um die oberen Muskelschichten voneinander zu lösen.
  \item \fett{Parmaschinken rollen:} mit wenig Öl in mehreren Bahnen längs zum Rücken vom Becken in Richtung Kopf und danach quer zum Rücken
\end{enumerate}

Nach diesem Schritt braucht ihr wieder mehr Öl. Es ist also Zeit zum Nach-Ölen.

\iftoggle{long}{
  \pagebreak
}{}
\subsection{Eine Rückenhälfte bearbeiten}

Nehmt euch zuerst eine Rückenhälfte komplett vor, danach die andere.
\iftoggle{long}{
  Bittet eure Partnerin, ihren Kopf von euch weg zu drehen, weil sie ansonsten die Schultermuskeln anspannt, die ihr massieren wollt.
}{}

Ihr sitzt dabei auf der Seite, die ihr massiert.

\subsubsection{Mitteltiefes und Tiefes schräg von der Seite}

Hier sitzt ihr \fett{schräg neben eurer Partnerin}.

\begin{oframed}
  \begin{enumerate}
    \item \fett{Streichungen mit Fingerflächen:} Legt beide Hände übereinander und streicht mit Druck mit den Fingerflächen fächerförmig schräg nach außen. Diese Streichungen sind recht kurz~-- maximal eine Handlänge.
    \item \fett{Daumenstreichungen:} Streicht mit beiden Daumen fächerförmig mit viel Druck nach außen. Die Daumen zeigen dabei in Richtung des Kopfes. Diese Streichungen sind sehr kurz~-- etwa eine halbe Handlänge. Achtung, manchen Menschen ist diese Technik zu intensiv.
  \end{enumerate}
\end{oframed}

\subsubsection{Schultern schräg von der Seite}

\begin{oframed}
  \begin{enumerate}
    \iftoggle{long}{
      \item \fett{Milchtritt/Tigerkrallen:} Krallt die Muskeln um die Schulterblätter.
      \item \fett{Unter dem Schulterblatt:} Hebt das Schulterblatt und den Ellenbogen eurer Partnerin an und stützt diese mit einem Arm (und bei Bedarf mit eurem Knie). Massiert dann mit der Fingerspitzen mit viel Druck unter dem Schulterblatt entlang.
    }{}
    \item \fett{Schultermuskeln kneten:} Greift kräftig rein und knetet den Kuchenteig. Diese Technik funktioniert ausnahmsweise besser, wenn ihr auf der gegenüberliegenden Seite sitzt. Eure Partnerin sollte dabei ihren Kopf etwas in Richtung Brust anwinkeln.
  \end{enumerate}
\end{oframed}

\pagebreak
\subsection{Nacken}

Bittet eure Partnerin, ihre Stirn auf ihren Hände abzustützen, sodass ihr Kopf gerade liegt.

\begin{description}
  \item[Nacken kneten:] Knetet sacht mit beiden Händen ihren Nacken, indem ihr den Daumen einer Hand links neben die Wirbelsäule legt und die Finger dieser Hand rechts (oder umgekehrt) und dann beide Hände abwechseln nach oben zieht.
  \item[Nacken ausstreichen:] Streicht beide Nackenseiten mit den Händen vom Kopf zu den Schultern aus.
\end{description}

\iftoggle{long}{
  \subsection{Drücken mit Knack}

  Kniet euch längs \fett{über eure Partnerin}.

  Legt eure \fett{Handballen links und rechts neben die Wirbelsäule}~-- zwischen den Schulterblättern weit oben. Bitte eure Partnerin, ganz tief einzuatmen und dann komplett auszuatmen~-- tiefer als sonst. Beim Ausatmen drückt ihr dann senkrecht auf den Rücken in Richtung Matratze. Wenn alle Luft raus ist, drückt noch einmal kurz kräftig nach. Wiederholt diesen Griff dreimal und wandert jedes Mal etwas weiter \fett{von oben nach unten}. Spätestens am Ende der Rippen der Rippen ist Schluss.

  Der Druck ist dabei sehr punktuell mit den Handballen parallel zur Wirbelsäule, nicht mit den Handflächen oder den Fingern.

  Sehr wichtig ist hier, dass ihr nicht auf die Wirbelsäule drückt und dass ihr nicht zu lange drückt.

  \pagebreak


  \subsection{Tiefes von unten}

  Hier kniet ihr wieder über den Oberschenkeln eurer Partnerin.

  \begin{enumerate}
    \item \fett{Daumendruck-Kreisen auf dem Rückenstrecker:} Setzt eure Daumenspitzen links und rechts an der Wirbelsäule an, macht mit Druck einen kleinen Kreis mit jedem Daumen, wandert mit wenig Druck ein kleines Stück nach oben und macht den nächsten Kreis.
    \item \fett{Daumenstreichungen am Illiosakralgelenk:} Streicht etwas oberhalb des Illiosakralgelenks (also oberhalb der Stelle, wo die Wirbelsäule den Beckenknochen berührt) fächerförmig mit den Daumenspitzen schräg nach außen/oben.
    \item \fett{Beckenkamm-Kreise mit den Daumen:} Macht mit den Daumen kleine Kreise am Beckenkamm. Beide Daumen beschreiben dabei gleichzeitig einen Weg vom der Wirbelsäule zur Seite, an der Wirbelsäule gespiegelt. An der unteren Hälfte der Kreise streicht ihr nach außen.
    \item \fett{Bahnen mit aufgestellten Fingern:} Formt mit offenen Fingern Tigerkrallen und streicht mit den Fingerspitzen der aufgestellten, angespannten Finger kräftig von der Hüfte in Richtung Kopf.
  \end{enumerate}
  \pagebreak
}{}


\iftoggle{long}{
  \subsection{Gesäß}

  \subsubsection{Ziehen}

  Hier kniet ihr über den Oberschenkeln eurer Partnerin.

  Ihr zieht \fett{mit je einer Hand pro Seite} nach oben (von der Unterlage weg). Eure Hände \fett{überkreuzen} sich dann und wandern zur anderen Seite herunter.

  \subsubsection{Symmetrische kreisförmige Streichungen mit den Handflächen}

  Hier kniet ihr weiterhin über den Oberschenkeln eurer Partnerin.

  Dabei massiert ihr \fett{mit je einer Hand pro Seite}.

  \begin{oframed}
    \fett{Kreis-Rundreise:}

    \begin{enumerate}
      \item auf Gürtelhöhe \fett{nach außen}
      \item seitlich \fett{nach unten}
      \item unten \fett{nach innen}
      \item wieder \fett{in Richtung Kopf}
    \end{enumerate}
  \end{oframed}

  \subsubsection{Kreisförmige Streichungen auf einer Seite}

  Hier massiert ihr \fett{mit beiden Händen auf einer Seite}. Dabei sind eure \fett{Hände versetzt zueinander}. Die Route ist dieselbe wie mit zwei Händen.

  \subsubsection{Kneten}

  Hier kniet ihr seitlich neben eurer Partnerin und arbeitet an der  die gegenüberliegenden Gesäßseite.

  \begin{description}
    \item [Greifen] mit Daumen und Fingern zusammen
    \item [Zickzack-Kneten] mit Daumen gegenüber den Fingern
  \end{description}
}{}

\iftoggle{long}{
  \pagebreak
  \subsection{Rücken: Punktuelle Druckmassage}

  Diese Technik bedarf einiger Übung, bis es mit dem Verspannungen-Ertasten klappt.

  \fett{Tastet den Rücken ab} und findet die Stellen heraus, an denen noch Verspannungen sitzen. Drückt dort jetzt \fett{mit den Fingerflächen der Mittel- und Ringfinger} beider Hände übereinander sehr langsam \fett{fest zwei, drei Atemzüge lang} hinein. Macht danach mit der nächsten Stelle weiter.

  Ihr könnt dabei auch einen eurer Ellenbogen auf eurem Beim oder der Unterlage abstützen.

  Achtet bei dieser Stelle darauf, dass ihr nicht auf die Sehnen oder Nerven drückt.


  \subsection{Beißmassage des Trapezmuskels}

  Falls eure Partnerin es mag (und ihr euch hinreichend nahe steht), \fett{beißt die Muskeln zwischen den Schultern und dem Hals} (also den oberen Teil des Trapezmuskels). Ihr könnt auch vorsichtig \fett{am Nacken knabbern}.

  Für diese Massage solltet ihr ein Massageöl benutzen, das neutral oder angenehm schmeckt und auf Pflanzenölbasis ist. (Mineralöle können zu Durchfall führen.)
}{}

\iftoggle{long}{
  \pagebreak
}{}
\subsection{Abschluss und nachher}
\subsubsection{Streichungen}

Bei diesen Streichungen habt ihr je eine Hand pro Rückenhälfte.

\begin{description}
  \iftoggle{long}{
    \item [Versetzte Kreise:] Streicht kräftig mit den Handflächen so, dass eine Hand auf der einen Seite der Wirbelsäule nach oben geht, während die andere an der anderen Körperseite nach unten geht.
  }{}
  \item [Fingernagel-Streichungen nach oben:] Streicht mit "`Katzenpfötchen"' \fett{gegen die Kratzrichtung} nach oben. Beide Hände bewegen sich dabei parallel. Macht dabei mehrere Bahnen, sodass ihr die komplette Rückenfläche abdeckt.
  \item [Fingernagel-Streichungen nach außen:] Streicht mit Katzenpfötchen (wieder entgegen der Kratzrichtung) in Bahnen von der Wirbelsäule nach außen zur Seite.
\end{description}


\subsubsection{Verabschieden}

Verabschiedet euch mit einem langen Hände-Auflegen von dem Rücken eurer Partnerin.

\subsubsection{Danach}
\fett{Hände ausschütteln und waschen.} Holt euch \fett{Feedback}. Genießt das entspannte, glückliche \fett{Lächeln} eurer Partnerin.
